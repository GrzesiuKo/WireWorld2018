\documentclass[a4paper,11pt]{article}

\usepackage[T1]{fontenc}
\usepackage[polish]{babel}
\usepackage[utf8]{inputenc}
\usepackage{lmodern}
\selectlanguage{polish}
\usepackage[top=2cm, bottom=2cm, left=3cm, right=3cm]{geometry}

\makeatletter
\newcommand{\linia}{\rule{\linewidth}{0.4mm}}
\renewcommand{\maketitle}{\begin{titlepage}
    \vspace*{2cm}
    \begin{center}\LARGE
    Politechnika Warszawska\\
    Wydział Elektryczny\\
    \end{center}
    \vspace{5cm}
    \noindent\linia
    \begin{center}
      \LARGE \textsc{\@title}
         \end{center}
     \linia
    \vspace{0.5cm}
    \begin{flushright}
    \begin{minipage}{5cm}
    \textit{Autorzy:}\\
    \normalsize \textsc{\@author} \par
    \end{minipage}
    \vspace{5cm}
     \end{flushright}
    \vspace*{\stretch{6}}
    \begin{center}
    \@date
    \end{center}
  \end{titlepage}%
}
\makeatother
\author{Grzegorz Kopyt\\
Daniel Sporysz}
\title{Specyfikacja Implementacyjna \\
"Automat Komórkowy"}

\usepackage{graphicx}
\begin{document}

\maketitle


\tableofcontents
\vspace{1cm}
\noindent\linia





\section{Diagram klas i pakietów}


\includegraphics[width=\textwidth]{Automat2}






\noindent\linia
\section{Opis klas i pakietów}



\subsection{GUI}
krótki opis klasy
\subsubsection{Pola}

\subsubsection{Metody}



\noindent\linia
\subsection{Logic}
krótki opis klasy
\subsubsection{Pola}

\subsubsection{Metody}



\noindent\linia

\subsection{IO}
krótki opis klasy
\subsubsection{Pola}

\subsubsection{Metody}



\noindent\linia

\section{Przepływ Sterowania}




\noindent\linia
\section{Testy klas i pakietów}


\subsection{GUI}
\begin{description}

\item[Scenariusze] \hfill
\begin{enumerate}
\item
\item 
\item
\item
\item 
\end{enumerate}

\item[Kryteria oceny poprawnej pracy] \hfill
\begin{enumerate}
\item 
\item
\item
\item 
\item 
\end{enumerate}

\end{description}


\end{document}



