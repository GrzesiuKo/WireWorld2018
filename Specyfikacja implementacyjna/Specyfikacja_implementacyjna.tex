\documentclass[a4paper,11pt]{article}

\usepackage[T1]{fontenc}
\usepackage[polish]{babel}
\usepackage[utf8]{inputenc}
\usepackage{lmodern}
\selectlanguage{polish}
\usepackage[top=2cm, bottom=2cm, left=3cm, right=3cm]{geometry}

\makeatletter
\newcommand{\linia}{\rule{\linewidth}{0.4mm}}
\renewcommand{\maketitle}{\begin{titlepage}
    \vspace*{2cm}
    \begin{center}\LARGE
    Politechnika Warszawska\\
    Wydział Elektryczny\\
    \end{center}
    \vspace{5cm}
    \noindent\linia
    \begin{center}
      \LARGE \textsc{\@title}
         \end{center}
     \linia
    \vspace{0.5cm}
    \begin{flushright}
    \begin{minipage}{5cm}
    \textit{Autorzy:}\\
    \normalsize \textsc{\@author} \par
    \end{minipage}
    \vspace{5cm}
     \end{flushright}
    \vspace*{\stretch{6}}
    \begin{center}
    \@date
    \end{center}
  \end{titlepage}%
}
\makeatother
\author{Grzegorz Kopyt\\
Daniel Sporysz}
\title{Specyfikacja Implementacyjna \\
"Automat Komórkowy"}

\usepackage{graphicx}
\begin{document}

\maketitle


\tableofcontents
\vspace{1cm}
\noindent\linia





\section{Diagram klas i pakietów}


\includegraphics[width=\textwidth]{Automat2}






\noindent\linia
\section{Pakiet GUI}

Zawiera dodatkową bibliotekę: jfoenix.

\subsection{Main}
Rozszerza klasę Application z javafx.
\subsubsection{Pola}
brak
\subsubsection{Metody}
\begin{itemize}
\item \textbf{main}

Standardowo wywołuje metodę launch.
\item \textbf{start}

Wczytuje plik MainScreen.fxml, przygotowuje całą scenę i wyświetla ją w wymiarach (800, 600).
\end{itemize}


\subsection{MainController}
Kontroler sceny MainScreen.fxml.
\subsubsection{Pola}
\begin{itemize}
\item GridPane gridPane
\item JFXButton load
\item JFXButton go
\item JFXButton pause
\item JFXButton halt
\item JFXButton clear
\item JFXButton color
\item Button figure1
\item Button figure2
\item Button figure3
\item Button figure4
\item Button figure5
\item Button figure6
\item JFXTextField path
\end{itemize}
\subsubsection{Metody}
\begin{itemize}
\item \textbf{loadFile}

Standardowo wywołuje metodę launch.
\item \textbf{goAnimation}

Uruchamia animacje wywołując metodę z klasy Animation.
\item \textbf{pauseAnimation}

Pauzeuje animacje metodą z klasy Animation.
\item \textbf{haltAnimation}

Powoduje powrót animacji do punktu początkowego.
\item \textbf{clear}

Zmienia kolor każdej komórki w tablicy na biały.
\item \textbf{colorMenu}

Wyświetla okno z pliku ColorMenu.fxml
\end{itemize}






\noindent\linia

\section{Pakiet FXML}
\begin{itemize}
\item MainScreen.fxml
\item ColorMenu.fxml
\end{itemize}
\noindent\linia

\section{Pakiet Logic}

\subsection{Logic}
krótki opis klasy
\subsubsection{Pola}

\subsubsection{Metody}



\noindent\linia
\section{Pakiet IO}
\subsection{IO}
krótki opis klasy
\subsubsection{Pola}

\subsubsection{Metody}



\noindent\linia

\section{Pakiet Board}

\subsection{BoardMaker}
Odpowiada za stworzenie tablicy obiektów klasy Rectangle. Tablica ta będzie służyła jako obszar edytowany przez użytkownika, a także będzie na niej wyświetlana animacja.
\subsubsection{Pola}

\subsubsection{Metody}




\subsection{Factory}
krótki opis klasy
\subsubsection{Pola}

\subsubsection{Metody}

\noindent\linia

\section{Pakiet Animation}

\subsection{Animation}
krótki opis klasy
\subsubsection{Pola}

\subsubsection{Metody}

\noindent\linia
\section{Przepływ Sterowania}




\noindent\linia
\section{Testy klas i pakietów}


\subsection{GUI}
\begin{description}

\item[Scenariusze] \hfill
\begin{enumerate}
\item
\item 
\item
\item
\item 
\end{enumerate}

\item[Kryteria oceny poprawnej pracy] \hfill
\begin{enumerate}
\item 
\item
\item
\item 
\item 
\end{enumerate}

\end{description}


\end{document}



