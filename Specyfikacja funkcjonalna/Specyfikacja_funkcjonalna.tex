\documentclass[a4paper,11pt]{article}

\usepackage[T1]{fontenc}
\usepackage[polish]{babel}
\usepackage[utf8]{inputenc}
\usepackage{lmodern}
\selectlanguage{polish}
\usepackage[top=2cm, bottom=2cm, left=3cm, right=3cm]{geometry}
\makeatletter
\newcommand{\linia}{\rule{\linewidth}{0.4mm}}
\renewcommand{\maketitle}{\begin{titlepage}
    \vspace*{2cm}
    \begin{center}\LARGE
    Politechnika Warszawska\\
    Wydział Elektryczny\\
    \end{center}
    \vspace{5cm}
    \noindent\linia
    \begin{center}
      \LARGE \textsc{\@title}
         \end{center}
     \linia
    \vspace{0.5cm}
    \begin{flushright}
    \begin{minipage}{5cm}
    \textit{Autorzy:}\\
    \normalsize \textsc{\@author} \par
    \end{minipage}
    \vspace{5cm}
     \end{flushright}
    \vspace*{\stretch{6}}
    \begin{center}
    \@date
    \end{center}
  \end{titlepage}%
}
\makeatother
\author{Grzegorz Kopyt\\
Daniel Sporysz}
\title{Specyfikacja Funkcjonalna \\
,,Automat Komórkowy - WireWorld''}
\usepackage{graphicx}

\begin{document}

\maketitle

\tableofcontents
\vspace{1cm}
\noindent\linia
\section{Opis działania}
Program jest implementacją automatu komórkowego opartego na regułach ,,gry w życie'' Johna Conwaya w wariancie ,,WireWorld''.

Za pomocą interfejsu graficznego program przedstawia zmiany pól jakie zachodzą na planszy zgodnie z zasadami gry.

Pracę programu można konfigurować na kilka sposobów: wczytać początkową konfigurację planszy z pliku graficznego, stworzyć własną planszę za pomocą narzędzi do edycji pól lub w sposób mieszany, czyli wczytując konfigurację z pliku i dalsza jego edycja za pomocą edytora programu.
Konfiguracji podlegają również metoda analizy planszy, jak również szybkość wyświetlania kolejnych przejść.

Analizę i modyfikację planszy można przerwać lub wznowić w każdej chwili pracy programu. Gdy generacja jest wstrzymana, użytkownik może ręcznie zmodyfikować planszę, zmienić sposób analizy planszy lub zapisać jej obecny stan do pliku graficznego.

Program umożliwia również ciągły zapis plansz do plików graficznych z których później możliwe jest stworzenie jednego pliku w formacie ,,GIF''.

\noindent\linia
\section{Funkcje}
Program oferuje następujące funkcje:
\begin{itemize}
\item Łatwa konfiguracja i obsługa pracy programu przez interfejs graficzny
\item Sterowane programem w czasie rzeczywistym - pauzowanie, wznawianie i ustawianie szybkości generowania kolejnych plansz
\item Odczyt planszy z pliku graficznego
\item Narzędzia do edycji planszy
\begin{itemize}
\item narzędzie ołówka, rysowania linii, gumka
\item wybór koloru
\end{itemize}
\item Określenie wymiarów planszy
\item Opcja wstawiania gotowych elementów z przygotowanej biblioteki obiektów WireWorld
\item Zapis aktualnego stanu planszy do pliku graficznego
\item Zapis ciągły generowanych pól do plików graficznych
\item Tworzenie plików .GIF
\item Czyszczenie planszy

\end{itemize}

\noindent\linia
\section{Obsługa}
Obsługa i konfiguracja pracy programu zachodzi przez interfejs graficzny.


\includegraphics[width=\textwidth]{GUI_WireWorld}

\noindent\linia
\section{Sytuacje wyjątkowe}
\begin{itemize}
\item Podano zbyt duże wymiary planszy
\item Plik graficzny zawiera planszę o zbyt dużych wymiarach
\item Podano złą ścieżkę do pliku graficznego
\item Podany plik nie jest plikiem graficznym
\item Program nie ma praw do odczytu plików 
\item Program nie ma praw do zapisu plików
\end{itemize}

\end{document}



